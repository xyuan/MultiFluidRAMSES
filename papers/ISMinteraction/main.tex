%% This is emulateapj reformatting of the AASTEX sample document
%%
%\documentclass[iop]{emulateapj-rtx4}
%\documentclass[iop]{aastex}

%\documentclass[oldversion]{aa}
\documentclass[referee,oldversion]{aa}


\usepackage{graphicx}
\usepackage{graphpap}
\usepackage{epsf}
\usepackage{amssymb}
\usepackage{color}

\usepackage{natbib}

\sloppy

%\hbadness 10000 \topmargin -1cm \oddsidemargin -.5truecm
%\evensidemargin=\oddsidemargin \textheight=23cm \textwidth=16.5cm

\def\FIG #1 #2 [#3] #4\par{%
%  \begin{figure} [ht]\begin{center}%
  \begin{figure}\begin{center}%
    \includegraphics*[#3]{#2}%
    \\
    \caption{#4}%
    \label{#1}%
  \end{center}\end{figure}%
}
\def\FIGG #1 #2 #3 [#4] #5\par{%
  \begin{figure*}[ht]\begin{center}%
%    \resizebox{\hsize}{!}{
               \includegraphics*[#4]{#2}
               \includegraphics*[#4]{#3}%}%
    \caption{#5}%
    \label{#1}%
   \end{center}\end{figure*}%
}
%
\def\FIGth #1 #2 #3 #4 [#5] #6\par{%
  \begin{figure*}[ht]\begin{center}%
        \includegraphics[#5]{#2} \\
        \includegraphics[#5]{#3}
        \includegraphics[#5]{#4}
        \caption{#6}
        \label{#1}
   \end{center}\end{figure*}
}
% 
\def\FIGfo #1 #2 #3 #4 #5 [#6] #7\par{%
  \begin{figure*}[ht]\begin{center}%
        \includegraphics[#6]{#2}
        \includegraphics[#6]{#3} \\
        \includegraphics[#6]{#4}
        \includegraphics[#6]{#5}
        \caption{#7}
        \label{#1}
   \end{center}\end{figure*}
}
% 
\def\FIGsi #1 #2 #3 #4 #5 #6 #7 [#8] #9\par{%
  \begin{figure*}[ht]\begin{center}%
        \includegraphics[#8]{#2}
        \includegraphics[#8]{#3}\\
        \includegraphics[#8]{#4}
        \includegraphics[#8]{#5}\\
        \includegraphics[#8]{#6}
        \includegraphics[#8]{#7}
        \caption{#9}
        \label{#1}
    \end{center}\end{figure*}
}

\def\rfig#1{Fig. \ref{#1}}

\def\etal{et al. }
\def\msun{M_\odot}
\def\Ek{E_{\mathrm{thr}}}
\def\neh{n_{e, > \Ek}}
\def\Ka{$K_{\alpha}$ }

\def\spex{{\sc spex}}
\def\rhoc{\rho_\mathrm{CSM}}

\newcommand{\dk}[1]{{\bf{\color{red}!D:} #1}}
\newcommand{\mt}[1]{\mathrm{#1}}
\newcommand{\sect}[1]{Section~\ref{#1}}
 
\def\lvm{\leavevmode\hbox to\parindent{\hfill}}
\def\req#1{(\ref{#1})}

\def\d{\partial}   
\def\dT{\partial T}
\def\dt{\partial t}
\def\du{\partial u}
\def\dr{\partial r}
\def\dm{\partial m}
\def\dP{\partial P}
\def\dE{\partial E}
\def\dS{\partial S}
\def\dX{\partial X}
\def\drho{\partial \rho}
\def\L{\left} 
\def\R{\right}
\def\eps{\varepsilon}
\def\kelv{{\rm K}}
\def\BE{\begin{equation}}
\def\EE{\end{equation}}
\def\BA{\begin{array}}
\def\EA{\end{array}}
\def\BAN{\begin{eqnarray*}}
\def\EAN{\end{eqnarray*}}
\def\apj{Astrophys. J.}
 
\def\la{\mathrel{\mathpalette\fun <}}
\def\ga{\mathrel{\mathpalette\fun >}}
\def\fun#1#2{\lower3.6pt\vbox{\baselineskip0pt\lineskip.9pt
\ialign{$\mathsurround=0pt#1\hfil##\hfil$\crcr#2\crcr\sim\crcr}}}
%\def\plotone#1{\epsfysize=8cm\epsffile[-182 144 392 718]{#1}}
\def\plotone#1{\epsfysize=15cm\epsffile{#1}}
\def\pn{\par\noindent}
\def\pnh{\par\vskip 0.5pc\noindent}
\def\pvh{\par\vskip 0.5pc}
\def\ergs{ergs s$^{-1}$ }
\def\gmc{g cm$^{-3}$}
\def\kms{km s$^{-1}$}
%\def\msun{\hbox{M}_\odot}
\def\msun{M_\odot}
\def\Msun{$\msun$}
\def\Rsun{$R_\odot$}
\def\fefsx{$^{56}$Fe}
\def\cofsx{$^{56}$Co}
\def\nifsx{$^{56}$Ni}
\def\e#1{$\times 10^{#1}$ }
\def\ee#1{$10^{#1}$ }
\def\lum#1{$L_{\rm #1}$}
\def\mas#1{$M_{\rm #1}$ }
\def\etal{et al. }
\def\ltsima{$\; \buildrel < \over \sim \;$}
\def\ltsim{\lower.5ex\hbox{\ltsima}}
\def\gtsima{$\; \buildrel > \over \sim \;$}
\def\gtsim{\lower.5ex\hbox{\gtsima}}
\def\Msunyr{~M_\odot~{\rm yr}^{-1}}

\def\Ecr{E_\mt{CR}}
\def\Pcr{P_\mt{CR}}
\def\PQ{PQ_\mt{CR}}
\def\qcr{q_\mt{CR}}
\def\kcr{\kappa_\mt{CR}}
\def\ecr{\epsilon_\mt{esc}}
\def\lcr{\lambda_\mt{esc}}
\def\gcr{\gamma_\mt{CR}}
\def\kcrd{\;\mt{cm}^2\mt{s}^{-1}}
 

%bibliography
\def\apj{ApJ}
\def\aap{A\&A}
\def\apjs{ApJ Supl.}
\def\aaps{A\&A Supl.}
\def\mnras{MNRAS}



\def\basedir{../../ramses/output}

\def\xilo{\basedir/bckreact_b5_n01_E1_M14_t3d3}
\def\xime{\basedir/weakxi4_b5_n01_E1_M14_t3d3}
\def\xihi{\basedir/seedxi5_b5_n01_E1_M14_t3d3}


\usepackage{txfonts}
%\usepackage{natbib}
\bibpunct{(}{)}{,}{a}{}{,}

\begin{document}

\title{Expansion of the SNR into ISM}
\titlerunning{SNRs expanding into ISM}

\author{D.~Kosenko\inst{1,2} \and A. Decourchelle\inst{1}}
   \institute{
CEA, Saclay, France
    \email{daria.kosenko@cea.fr}
         \and
SAI, Moscow, 119992, Russia
             }

\authorrunning{Kosenko et al.}
\date{Received ...; accepted ... }

\abstract{
To study the impact on ISM from SNR. Core-Collapse and Thermonuclear. CR acceleration on/off. Expanding into CSM/ISM. 
}

\keywords{}

%\maketitle

%\tableofcontents

%\section{Introduction}
%\section{Method}
%\section{Initial condition}


%\FIGsi profiles  \xilo/radius_t.pdf  \xime/radius_t.pdf  \xihi/radius_t.pdf  \xilo/Pc_Ptot_t.pdf  \xime/Pc_Ptot_t.pdf  \xihi/Pc_Ptot_t.pdf   \xilo/r_tot_t.pdf   \xime/r_tot_t.pdf  \xihi/r_tot_t.pdf  [width=0.1\hsize,angle=0]  Profiles

%\section{Numerical models}
  \begin{figure*}[ht]\begin{center}%
        \includegraphics*[width=0.29\hsize]{\xilo/radius_t.pdf}
        \includegraphics*[width=0.29\hsize]{\xilo/Pc_Ptot_t.pdf}
        \includegraphics*[width=0.29\hsize]{\xilo/r_tot_t.pdf}\\
        \includegraphics*[width=0.29\hsize]{\xime/radius_t.pdf}
        \includegraphics*[width=0.29\hsize]{\xime/Pc_Ptot_t.pdf}
        \includegraphics*[width=0.29\hsize]{\xime/r_tot_t.pdf}\\
        \includegraphics*[width=0.29\hsize]{\xihi/radius_t.pdf}
        \includegraphics*[width=0.29\hsize]{\xihi/Pc_Ptot_t.pdf}
        \includegraphics*[width=0.29\hsize]{\xihi/r_tot_t.pdf}\\
        \caption{Temporal evolution of the radii (left column), normalized CR pressure (middle column), total compression ratio (right column). Top row: $\lg{\xi} = 3.5$, middle row: $\lg{\xi} = 4.0$, bottom row: $\lg{\xi} = 5.0$, where $\xi$ defines an effective injection momentum through eq.~(25) in Blasi et al. (2005).  Legend: FS --- forward shock, RS --- reverse shock.}
        \label{evol1}
    \end{center}\end{figure*}
%  \begin{figure*}[ht]\begin{center}%
%        \includegraphics*[width=0.4\hsize]{w7_300_2d25_3d3_k1d25qcr00fs_430_prof} 
%        \includegraphics*[width=0.4\hsize]{w7_300_RGw01_3d3_k1d25qcr00fs_430_prof} 
%        \includegraphics*[width=0.4\hsize]{w7_300_AGB01t10w_3d3_k1d25qcr00fs_430_prof}
%        \includegraphics*[width=0.4\hsize]{w7_300_AGB01t25w_3d3_k1d25qcr00fs_430_prof}
%        \caption{Modeled  ray-traced radial brightness profiles which correspond to Fig.~\ref{winds01} and ~\ref{windsp1} at the age 430 yr,  $n_0 = 0.1\;\mt{cm}^{-3}$. From left to right, then from top to bottom: homogeneous ISM, RG wind,  AGB wind ($\dot{M} = 10^{-5}\;M_\odot/yr$,  t = 0.1 Myr), AGB wind (t = 0.25 Myr)}
%        \label{windprof1}
%    \end{center}\end{figure*}

%\section{Discussion}

%\begin{table*}[htdp]
%\caption{default}
%\begin{center}
%\begin{tabular}{|c|c|}
%
%\end{tabular}
%\end{center}
%\label{default}
%\end{table}%


%\section{Conclusion}
%\label{conclusion}

\bibliographystyle{aa}
%\bibliography{mainbib}

\end{document}
